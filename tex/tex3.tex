\documentclass[dvipdfmx]{jsarticle}


\usepackage{tcolorbox}
\usepackage{color}
\usepackage{listings, plistings}

%% ノート/latexメモ
%% http://pepper.is.sci.toho-u.ac.jp/pepper/index.php?%A5%CE%A1%BC%A5%C8%2Flatex%A5%E1%A5%E2

% Java
\lstset{% 
  frame=single,
  backgroundcolor={\color[gray]{.9}},
  stringstyle={\ttfamily \color[rgb]{0,0,1}},
  commentstyle={\itshape \color[cmyk]{1,0,1,0}},
  identifierstyle={\ttfamily}, 
  keywordstyle={\ttfamily \color[cmyk]{0,1,0,0}},
  basicstyle={\ttfamily},
  breaklines=true,
  xleftmargin=0zw,
  xrightmargin=0zw,
  framerule=.2pt,
  columns=[l]{fullflexible},
  numbers=left,
  stepnumber=1,
  numberstyle={\scriptsize},
  numbersep=1em,
  language={Java},
  lineskip=-0.5zw,
  morecomment={[s][{\color[cmyk]{1,0,0,0}}]{/**}{*/}},
  keepspaces=true,         % 空白の連続をそのままで
  showstringspaces=false,  % 空白字をOFF
}
%\usepackage[dvipdfmx]{graphicx}
\usepackage{url}
\usepackage[dvipdfmx]{hyperref}
\usepackage{amsmath, amssymb}
\usepackage{itembkbx}
\usepackage{eclbkbox}	% required for `\breakbox' (yatex added)
\usepackage{enumerate}
\usepackage{setspace}
\usepackage{multicol}
\usepackage[default]{cantarell}
\usepackage[T1]{fontenc}
\fboxrule=0.5pt
\parindent=1em
\begin{document}

%\anaumeと入力すると穴埋め解答欄が作れるようにしてる。\anaumesmallで小さめの穴埋めになる。
\newcounter{mycounter} % カウンターを作る
\setcounter{mycounter}{0} % カウンターを初期化
\newcommand{\anaume}[1][]{\refstepcounter{mycounter}{#1}{\boxed{\phantom{aa}\themycounter \phantom{aa}}}} %穴埋め問題の空欄作ってる。
\newcommand{\anaumesmall}[1][]{\refstepcounter{mycounter}{#1}{\boxed{\tiny{\phantom{a}\themycounter \phantom{a}}}}}%小さい版作ってる。色々改造できる。

%% 修正時刻: Sun Oct 31 10:15:15 2021


\section{ダッシュボードに設定画面を作る}

ダッシュボードに設定画面を作ってみる。今回は、「設定」タブに ``onclick plugin'' の
項目を追加してみる。

onclick-plugin.php と同じフォルダに \emph{onclick-plugin-menu.php} を作成し、
以下の内容とする。

\begin{lstlisting}[caption=onclick-plusin-menu.php, language=PHP]
<?php
function onclick_plugin_menu() {
  add_options_page (
    'onclick plugin 設定',        // 管理ページのタイトル
    'onclick plugin',             // 管理メニュー名
    'manage_options',             // 管理ページのコンテンツを表示するのに必要な権限
    'onclick-plugin-menu.php',    // 管理ページのコンテンツを表示する phpファイル
    'onclick_plugin_admins_page'  // 管理ページのコンテンツを表示する関数
  );
}
add_action('admin_menu', 'onclick_plugin_menu');

function onclick_plugin_admins_page() {
  ?>
  <div class="wrap">
    <h2>onclick plugin 設定</h2>
    <p>このショートコードをコピーしてください</p>
    <input type="text" onfocus="this.select()"
           style="font-size: 24px" value="[insert_onclick]"/>
  </div>
<?php 
}
\end{lstlisting}

そして、このファイルを onclick-plugin.php で読み込む。

\begin{lstlisting}[caption=onclick-plugin.php]
... (略) ...
 * Author: Seiichi Nukayama
 */
 
require_once('onclick-plugin-menu.php');     // <==

function add_somefiles() {
 ... (略) ...
\end{lstlisting}

これで、ダッシュボードの「設定」タブに ``onclick plugin'' という項目で
できている。

それをクリックすると、''onclick plugin 設定''画面が開く。

\vspace\baselineskip

(\textgt{参考}) \href{https://oxynotes.com/?p=9321}{本気で作りたい人向け、WordPressプラグインの作成方法}


\end{document}

%% 修正時刻: Sat May  2 15:10:04 2020


%% 修正時刻: Wed 2022/11/09 17:36:151
