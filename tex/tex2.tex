\documentclass[dvipdfmx]{jsarticle}


\usepackage{tcolorbox}
\usepackage{color}
\usepackage{listings, plistings}

%% ノート/latexメモ
%% http://pepper.is.sci.toho-u.ac.jp/pepper/index.php?%A5%CE%A1%BC%A5%C8%2Flatex%A5%E1%A5%E2

% Java
\lstset{% 
  frame=single,
  backgroundcolor={\color[gray]{.9}},
  stringstyle={\ttfamily \color[rgb]{0,0,1}},
  commentstyle={\itshape \color[cmyk]{1,0,1,0}},
  identifierstyle={\ttfamily}, 
  keywordstyle={\ttfamily \color[cmyk]{0,1,0,0}},
  basicstyle={\ttfamily},
  breaklines=true,
  xleftmargin=0zw,
  xrightmargin=0zw,
  framerule=.2pt,
  columns=[l]{fullflexible},
  numbers=left,
  stepnumber=1,
  numberstyle={\scriptsize},
  numbersep=1em,
  language={Java},
  lineskip=-0.5zw,
  morecomment={[s][{\color[cmyk]{1,0,0,0}}]{/**}{*/}},
  keepspaces=true,         % 空白の連続をそのままで
  showstringspaces=false,  % 空白字をOFF
}
%\usepackage[dvipdfmx]{graphicx}
\usepackage{url}
\usepackage[dvipdfmx]{hyperref}
\usepackage{amsmath, amssymb}
\usepackage{itembkbx}
\usepackage{eclbkbox}	% required for `\breakbox' (yatex added)
\usepackage{enumerate}
\usepackage{setspace}
\usepackage{multicol}
\usepackage[default]{cantarell}
\usepackage[T1]{fontenc}
\fboxrule=0.5pt
\parindent=1em
\begin{document}

%\anaumeと入力すると穴埋め解答欄が作れるようにしてる。\anaumesmallで小さめの穴埋めになる。
\newcounter{mycounter} % カウンターを作る
\setcounter{mycounter}{0} % カウンターを初期化
\newcommand{\anaume}[1][]{\refstepcounter{mycounter}{#1}{\boxed{\phantom{aa}\themycounter \phantom{aa}}}} %穴埋め問題の空欄作ってる。
\newcommand{\anaumesmall}[1][]{\refstepcounter{mycounter}{#1}{\boxed{\tiny{\phantom{a}\themycounter \phantom{a}}}}}%小さい版作ってる。色々改造できる。

%% 修正時刻: Sun Oct 31 10:15:15 2021


\section{プラグインに変更する}

index.html をもとに プラグインを作っていく。

\subsection{プラグインファイルをつくる}

以下のように、プラグインのファイルを作る。
プラグイン名がわかるように名前をつける。
index.html と同じ場所に置く。

\vspace{3mm}
\framebox[4cm][c]{\textsf{onclick-plugin.php}}
\vspace{3mm}


そして、ファイルの先頭部分に以下の記述をする。

\begin{lstlisting}[caption=onclick-plugin.php]
<?php
/*
 * @wordpress-plugin 
 * Plugin Name: Onclick Plugin 
 * Description: 'onclick'のテスト。ショートコードは '[insert_onclick]' 。
 * Version: 1.0 
 * Author: Seiichi Nukayama
 */
\end{lstlisting}

\begin{enumerate}
 \item プラグインであることを WordPress に伝えている。
 \item ダッシュボードのプラグイン一覧に表示される。
 \item プラグイン一覧に表示される。ショートコードもここに書いておく。
 \item プラグインの管理上、必要。
 \item 作者名も書いておく。
\end{enumerate}

\subsection{各種スクリプトファイルの読込指定}

スタイルシート(onclick.css) や JavaScriptファイル(onclick.js) を読み込ませるための
記述が以下である。

\begin{lstlisting}[caption=onclick-plugin.php]
function add_somefiles() {
  wp_enqueue_script('onclick', plugins_url('js/onclick.js', __FILE__), array('jquery'), '1.0', true);
  wp_enqueue_style('onclick', plugins_url('css/onclick.css', __FILE__));
}
add_action('wp_enqueue_scripts', 'add_somefiles');
\end{lstlisting}

ここでは、\textsf{add\_somefiles} という名前の関数を定義し、それを \textsf{add\_action()}関数で
読み込んでいる。そして、それを \textsf{wp\_enqueue\_scripts}というアクションフックに登録している。

wp\_enqueue\_scripts というアクションフックに登録しておけば、適切なタイミングで JavaScriptを読み込ん
でくれるのである。

jsフォルダにある \textsf{onclick.js} は、\fbox{\textsf{wp\_enqueue\_scrpit()}} という関数で読み込ませる
ことができる。この関数は引数を5個もっている。

\vspace{3mm}
\begin{tabular}{|l|} \hline
\verb!wp_enqueue_script('onclick', plugins_url('js/onclick.js', __FILE__),! \\
\verb!                   array('jquery'), '1.0', true)! \\ \hline
\end{tabular}

\begin{enumerate}
 \item 'onclick\_js' \\
       ここで読み込ませる onclick.js のハンドル名。スクリプトに \verb!id="onclick-js"! として出力される。
 \item plugins\_url('js/onclick.js', \_\_FILE\_\_) \\
       読み込むファイルを指定。\textsf{plugins\_url()} を使うことで、プラグインのフォルダを指定できる。
       \textsf{\_\_FILE\_\_} 指定により、絶対パスを取得でき、それを親ディレクトリとして 'js/onclick.js'
       を指定できる。
 \item array('jquery') \\
       このスクリプトが依存するファイルを指定できる。ここでは jQuery を指定している。
       jQuery は WordPress がデフォルトで読み込んでくれているので、それを使うという意味である。
 \item '1.0' \\
       このバージョン番号はこのスクリプトが読み込まれるときに URL文字列にクエリ文字列として出力されるので、
       スクリプトファイルのバージョンを表すものとして使える。指定すべきである。
 \item true \\
       これを true にすると、Body の終了タグの直前にスクリプトを読み込んでくれる。
\end{enumerate}


\vspace{3mm}
\begin{tabular}{|l|} \hline
 \verb!wp_enqueue_style('onclick', plugins_url('css/onclick.css', __FILE__), array(),! \\
 \verb!                 '1.0');! \\ \hline
\end{tabular}

\begin{enumerate}
 \item 'onclick' \\
       スクリプトに \verb!id="onclick-css"! と出力される。
 \item plugins\_url('css/onclick.css', \verb!__FILE__!) \\
       onclick.css を プラグインフォルダ(絶対パス)に続けて読み込ませる。
 \item array() \\
       このスタイルシートの前に読み込むべきスタイルシートを指定できる。無ければ空の配列を記述しておく。
 \item '1.0' \\
       スクリプトの記述にクエリ文字列としてバージョンを記述できる。
\end{enumerate}


\subsection{画面出力部分}

続けて画面出力部分を記述する。

\begin{lstlisting}[caption=onclick-plugin.php]
function go_test() {
 ob_start();
?>
 <section>
   <button id="start">クリックしてね</button>
   <div id="area">
     <img id="close" src="<?php echo plugins_url('img/close.gif', __FILE__); ?>" alt="close">
     <p>これはクリックすると、文字列を表示するだけのシンプルなプラグインです。<br>
        プラグインの勉強のために作成しました。</p>
   </div>
 </section>
 <?php
 return ob_get_clean();
}
add_shortcode('insert_onclick', 'go_test');
\end{lstlisting}

\begin{description}
 \item[1行目] function go\_test() \\
       関数名を定義。この関数名は 15行目で 'insert\_onclick' というショートコードと結びつけている。
 \item[2行目] ob\_start() \\
       以下に記述する HTMLをすぐに出力せずに バッファリングする。
 \item[7行目] <?php echo plugins\_url('/img/close.gif', \verb!__FILE__!); ?> \\
       close.gif を plugins\_url() を使ってディレクトリ指定している。
 \item[13行目] return \verb!ob_get_clean()! \\
       ここでバッファリングしていた文字列を出力する。そのことで、'insert\_onclick' という
       ショートコードを記述した部分に出力されるのである。
\end{description}

\subsection{コードをまとめると\dots\dots}

\begin{lstlisting}[caption=onclick-plugin.php]
<?php
/*
 * @wordpress-plugin                                                      // <1>
 * Plugin Name: Onclick Plugin                                            // <2>
 * Description: 'onclick'のテスト。ショートコードは '[insert_onclick]' 。 // <3>
 * Version: 1.0                                                           // <4>
 * Author: Seiichi Nukayama                                               // <5>
 */

function add_somefiles() {
  wp_enqueue_script('onclick', plugins_url('js/onclick.js', __FILE__), array('jquery'), '1.0', true);
  wp_enqueue_style('onclick', plugins_url('css/onclick.css', __FILE__));
}
add_action('wp_enqueue_scripts', 'add_somefiles');
 
function go_test() {
 ob_start();
?>
 <section>
   <button id="start">クリックしてね</button>
   <div id="area">
     <img id="close" src="<?php echo plugins_url('img/close.gif', __FILE__); ?>" alt="close">
     <p>これはクリックすると、文字列を表示するだけのシンプルなプラグインです。<br>
        プラグインの勉強のために作成しました。</p>
   </div>
 </section>
 <?php
 return ob_get_clean();
}
add_shortcode('insert_onclick', 'go_test');
\end{lstlisting}


\subsection{\$ を jquery に変える}

最後に、JavaScriptのコード ``onclick.js'' を少し修正する。
というのは、WordPressでは ``\$'' は
使えないのである。``\$''を \textsf{jQuery} に変更する。

\begin{lstlisting}[caption=onclick.js]
'use strict';

jQuery(function () {
  jQuery('#start').on('click', function() {
    jQuery('#area').css('display', 'block');
    jQuery('#start').css('display', 'none');
  });

  jQuery('#close').on('click', function() {
    jQuery('#area').css('display', 'none');
    jQuery('#start').css('display', 'block');
  });
});
\end{lstlisting}


これでできた。

index.html は不要なので、削除する。フォルダ構成は、以下。

\begin{spacing}{0.8}
\begin{verbatim}
./onclick-plugin
 ├── css/
 │   └── onclick.css
 ├── img/
 │   └── close.gif
 ├── js/
 │   └── onclick.js
 └── onclick-plugin.php
\end{verbatim}
\end{spacing}

この onclick-plugin フォルダを zip形式で圧縮する。


これを WordPressのダッシュボードで
 ``プラグイン'' -- ``新規追加'' -- ``プラグインのアップロード''
を選択し、''参照''ボタンでアップロードすればよい。




\end{document}

%% 修正時刻: Sat May  2 15:10:04 2020


%% 修正時刻: Sat Nov 13 11:29:47 2021
