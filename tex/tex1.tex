\documentclass[dvipdfmx]{jsarticle}

\include{begin}

\section{簡単なWebアプリを作成する}

以下のような、簡単なWebアプリを作成してみた。

\begin{lstlisting}[caption=index.html]
 <!doctype html>
 <html lang="ja">
   <head>
     <meta charset="utf-8"/>
     <title>Click me</title>
     <link rel="stylesheet" href="css/onclick.css"/>
   </head>
   <body>
     <h1>Click me</h1>
     <section>
       <button id="start">クリックしてね</button>
       <div id="area">
         <img id="close" src="./img/close.gif" alt="close"/>
         <p>これはクリックすると、文字列を表示するだけのシンプルなプラグインです。<br/>
         プラグインの勉強のために作成しました。</p>
       </div>
     </section>
     <script src="https://ajax.googleapis.com/ajax/libs/jquery/3.6.0/jquery.min.js"></script>
     <script src="js/onclick.js"></script>
   </body>
 </html>
\end{lstlisting}

\begin{multicols}{2}
\begin{lstlisting}[caption=onclick.js]
'use strict';

$(function () {
  $('#start').on('click', function() {
    $('#area').css('display', 'block');
    $('#start').css('display', 'none');
  });

  $('#close').on('click', function() {
    $('#area').css('display', 'none');
    $('#start').css('display', 'block');
  });
});
\end{lstlisting}


\begin{lstlisting}[caption=onclick.css]
@charset "UTF-8";

#area {
  display: none;
}

#start {
  cursor: pointer;
}

#close {
  cursor: pointer;
}
\end{lstlisting}
\end{multicols}

\vspace{3mm}
\includegraphics[width=4cm]{img/app01.png}
\vspace{3mm}

\vspace{3mm}
\hspace{10mm}
\includegraphics[width=5mm]{img/arrow-down.png}
\vspace{3mm}


\vspace{3mm}
\includegraphics[width=10cm]{img/app02.png}
\vspace{3mm}

フォルダ構成は、以下のとおり。

\begin{spacing}{0.8}
\begin{verbatim}
./onclick-plugin
├── css/
│   └── onclick.css
├── img/
│   └── close.gif
├── index.html
└── js/
    └── onclick.js
\end{verbatim}
\end{spacing}

    


\include{end}

%% 修正時刻: Sun Oct 31 10:46:56 2021
